\chapter{Resumen}
El presente proyecto terminal presenta una propuesta para implementar un sistema de monitoreo remoto para ayudar en la conservación de animales utilizando una red inalámbrica de sensores  (WSN), un enjambre de drones y un servidor web con el propósito de ofrecer una alternativa a los métodos tradicionales que presentan diversas limitaciones y desafíos, proponiendo que el enjambre de drones recolecte la información obtenida por la WSN. Eligiendo la estrategia de vuelo de los drones a partir de dos esquemas de recolección de datos, los cuales serán comparados en términos de consumo energético y tiempo de vuelo. Durante la ejecución de este trabajo, se logró establecer una colaboración con investigadores del INECOL, quienes aceptaron participar activamente en la iniciativa para monitorear al venado de cola blanca. Presentando, por último, el análisis y diseño propuestos así como algunos resultados preliminares.
\\ \\
\noindent \textit{\textbf{Palabras Clave:} Red Inalámbrica de Sensores (WSN), enjambre de drones, monitoreo de fauna, telemetría, consumo energético.}  \\



%%%%%%fin del archivo

\endinput 